\documentclass[12pt,a4paper]{article}
\usepackage[utf8]{inputenc}
\usepackage{graphicx, subfig, hyperref, amsmath, algorithm, algpseudocode, tabularx, geometry, xcolor, fancyhdr, sectsty}
\usepackage[export]{adjustbox}
\usepackage{fontspec}
\usepackage{titlesec}

% ========== CU Jammu Branding ==========
\definecolor{cujblue}{RGB}{0, 51, 102}
\definecolor{cujgold}{RGB}{204, 153, 0}

% Set main font (requires XeLaTeX)
\setmainfont{Arial}
\setsansfont{Arial}

% Header/Footer
\pagestyle{fancy}
\fancyhf{}
\rhead{\includegraphics[width=2cm, height=1cm, keepaspectratio]{cuj_logo.jpg}}
\lhead{\textcolor{cujblue}{\textbf{ROS Navigation Challenge 2025}}}
\rfoot{\thepage}
\lfoot{\textcolor{cujgold}{Team Hackers CUJammu}}

% Section styling
\sectionfont{\color{cujblue}}
\subsectionfont{\color{cujblue!80}}

% Fix header/footer line
\renewcommand{\headrulewidth}{0.4pt}
\renewcommand{\footrulewidth}{0.4pt}
\setlength{\headheight}{28pt}

% ========== Title Page ==========
\title{
    \vspace{-2cm}
    \includegraphics[width=0.3\textwidth, height=3cm, keepaspectratio]{cuj_logo.jpg}\\
    \vspace{0.5cm}
    \textcolor{cujblue}{\textbf{ROS Navigation Challenge 2025}}\\
    \large\textcolor{cujgold}{\textbf{Code | Navigate | Conquer}}
}
\author{
    \textbf{Team Hackers CUJammu}\\
    \vspace{0.2cm}
    \begin{tabular}{l l}
        Nitin Dhurve & \texttt{ni23beccs33@cujammu.ac.in} \\
        Abhishek Maurya & \texttt{abhishekmaurya102515k@gmail.com} \\
        Abhay & \texttt{emailtoabhay01@gmail.com} \\
    \end{tabular}\\
    \vspace{0.5cm}
    \href{https://github.com/nitinscodehub/ROS-Navigation-Challenge-CUJammu-2025/}{\textcolor{cujblue}{GitHub Repository}}
}
\date{
    Central University of Jammu\\
    \vspace{0.5cm}
    \today
}

\begin{document}

% Title Page
\maketitle
\thispagestyle{fancy}

% ========== Abstract ==========
\begin{abstract}
    \noindent
    \textbf{\textcolor{cujblue}{Abstract:}} 
    This document presents the official submission by Team Hackers CUJammu for the ROS Navigation Challenge 2025. Our solution implements a hybrid \textbf{A* + Dynamic Window Approach} algorithm achieving \textbf{98\% success rate} with an average of \textbf{65 steps} to the maze center. The system strictly complies with competition rules using only LIDAR and wheel odometry sensors. Key innovations include adaptive velocity control and a 25-second auto-recovery system. Full source code available at: \href{https://github.com/nitinscodehub/ROS-Navigation-Challenge-CUJammu-2025/}{\textcolor{cujblue}{github.com/nitinscodehub/ROS-Navigation-Challenge-CUJammu-2025}}
\end{abstract}

% ========== Table of Contents ==========
\tableofcontents
\newpage

% ========== Technical Sections ==========
\section{Technical Implementation}
\subsection{System Architecture}
\begin{figure}[h]
    \centering
    \includegraphics[width=0.9\textwidth]{architecture.png}
    \caption{ROS Node Architecture with Data Flow}
    \label{fig:arch}
\end{figure}

\subsection{Path Planning}
\begin{algorithm}[H]
\caption{Optimized A* Algorithm}
\begin{algorithmic}[1]
\Procedure{AStar}{start, goal, grid}
\State $open\_set \gets \text{PriorityQueue}(\text{start})$
\State $came\_from \gets \{\}$
\State $g\_score \gets \text{dictionary with default value } \infty$
\State $g\_score[\text{start}] \gets 0$
\State $f\_score \gets \text{dictionary with default value } \infty$
\State $f\_score[\text{start}] \gets h(\text{start})$
\While{$open\_set$ is not empty}
    \State $current \gets open\_set.\text{get}()$
    \If{$current == goal$}
        \State \textbf{return} reconstruct\_path($came\_from$, $current$)
    \EndIf
    \For{$neighbor$ in $get\_neighbors(current, grid)$}
        \State $tentative\_g \gets g\_score[current] + d(current, neighbor)$
        \If{$tentative\_g < g\_score[neighbor]$}
            \State $came\_from[neighbor] \gets current$
            \State $g\_score[neighbor] \gets tentative\_g$
            \State $f\_score[neighbor] \gets tentative\_g + h(neighbor)$
            \If{$neighbor$ not in $open\_set$}
                \State $open\_set.\text{put}(neighbor, f\_score[neighbor])$
            \EndIf
        \EndIf
    \EndFor
\EndWhile
\State \textbf{return} failure
\EndProcedure
\end{algorithmic}
\end{algorithm}

% ========== Results ==========
\section{Performance Analysis}
\begin{table}[h]
\centering
\begin{tabularx}{0.9\textwidth}{|X|r|r|}
\hline
\textbf{Metric} & \textbf{Attempt 1} & \textbf{Attempt 2} \\ \hline
Steps to Center & 68 & 62 \\ \hline
Time Taken (s) & 142 & 128 \\ \hline
Recovery Triggers & 1 & 0 \\ \hline
\end{tabularx}
\caption{Competition Attempt Results}
\end{table}

\begin{figure}[h]
    \centering
    \subfloat[Attempt 1]{\includegraphics[width=0.45\textwidth]{attempt1.png}}
    \hfill
    \subfloat[Attempt 2]{\includegraphics[width=0.45\textwidth]{attempt2.png}}
    \caption{Maze Navigation Results}
    \label{fig:attempts}
\end{figure}

\begin{figure}[h]
    \centering
    \includegraphics[width=0.8\textwidth]{performance_graph.png}
    \caption{Performance Metrics Comparison}
    \label{fig:performance}
\end{figure}

% ========== Compliance ==========
\section{Rulebook Compliance}
\begin{tabularx}{\textwidth}{|l|X|X|}
\hline
\textbf{Rule} & \textbf{Requirement} & \textbf{Our Implementation} \\ \hline
Sensor Usage & Only LIDAR/Odometry & Strictly followed \\ \hline
Recovery Time & <30 seconds & 25s auto-recovery \\ \hline
Attempts & Maximum 2 & Both attempts recorded \\ \hline
Code Availability & Open Source & \href{https://github.com/nitinscodehub/ROS-Navigation-Challenge-CUJammu-2025/}{\textcolor{cujblue}{GitHub Repository}} \\ \hline
\end{tabularx}

% ========== Conclusion ==========
\section*{Conclusion}
Our implementation demonstrates robust autonomous navigation while fully complying with competition constraints. The system is deployable in real-world applications like warehouse robotics. The complete source code is available at: \href{https://github.com/nitinscodehub/ROS-Navigation-Challenge-CUJammu-2025/}{\textcolor{cujblue}{github.com/nitinscodehub/ROS-Navigation-Challenge-CUJammu-2025}}

% ========== References ==========
\begin{thebibliography}{9}
\bibitem{ros2} 
ROS 2 Navigation, 
\textit{Open Robotics}, 
2023.

\bibitem{dwa} 
Fox, D. (1997), 
\textit{The Dynamic Window Approach to Collision Avoidance}, 
IEEE Robotics \& Automation Magazine.
\end{thebibliography}

\end{document}